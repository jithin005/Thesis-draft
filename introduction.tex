\chapter{Introduction}


Android is leapt to 87.6\% of Smartphone OS Market Share, Q2 2016 \cite{OS}.
Almost 1.4 billion smartphones were bought all over the world in 2015 which is a 10\% increase from the previous year \cite{Symantec}.
As the increase in number of mobile devices increase, which shows clearly that mobile devices are replacing the personal computers at home and work place.
Most of them rely on smartphones and tablets for any internet-related work from web surfing to e-commerce transaction and online banking.
Five out of the six new phones were running Android. 
The reason why Android platform is popular can include:
\begin{itemize}
    \item Global partnerships and large installed base 
    \item Powerful development framework
    \item Open marketplace for distributing your apps
\end{itemize}


However, there is a side effect with this prosperity — mobile threat. 
Mobile threats are defined using three categories: malware, chargeware, and adware. 
In the past year, malware grow substantially in the world.  
Many people all over the world may suffer from such malware attacks without protection and prevention.
Nevertheless, these above reasons above cannot solve in a short time, so we need some techniques to detect malicious apps and protect our devices. 
The best technique appropriate for most people is automatic analysis, which people don’t have any knowledge of Android can also discriminate between malicious apps and healthy apps. 
\nocite{*}